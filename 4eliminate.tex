
\subsection{Eliminate earth orbiting satellites} \label{sec:4eliminate}

Rubin does not publish alerts for streaks associated with artificial satellites.
A subset of streaks, potentially consistent with Earth-orbiting satellites or Solar System objects,
will be determined by comparison with an appropriate catalog or catalogs at the \gls{USDF}.
Any object that corresponds to an object in this catalog will not be released in the  prompt
alerts or stored in the prompt alert database.
Any catalog that SLAC creates noting these corresponding identifications will be held at the OFFICIAL USE ONLY level within the \gls{USDF} at \gls{SLAC} and have appropriate export-controls and \gls{FOIA} exemptions.
Additionally no streaks will be  alerted on with a streak length greater than 10 deg/day relative to sidereal.
In the rare event an unknown solar system object source with a streak  length greater than 10 deg/day potentially corresponds with an Earth-impacting asteroid, SLAC shall implement a method to send any candidate  impactors to the Minor Planet Center during the embargo period, complying with encryption standards for data in transit during the embargo  period.

The estimated cost for this is based mainly on FTE - we do not consider the need for extra hardware unless the agencies wish for hardware encryption for the few occasions we need to send a secure message to \gls{MPC}.
An estimate is given in \tabref{tab:eliminate}. This also includes the current value in the operations plan.
The cost of delaying the data in an encrypted store is already covered in \secref{sec:3delay}

\tiny \begin{longtable} {|l|r|r|r|} \caption{The Alert Vetting System is all FTE cost - apart from unknown hardware at LLNL. \label{tab:eliminate}}\\ 
\hline 
\textbf{Description}&\textbf{Cost}&\textbf{Count}&\textbf{Total} \\ \hline
{FTE per year}&{\$500,000.00}&{2.5}&{\$1,250,000} \\ \hline
{Mission years}&{}&{10}&{\$12,500,000} \\ \hline
{Pre operations years}&{}&{3}&{\$3,750,000} \\ \hline
{Front end server}&{\$20,000.00}&{2}&{\$40,000} \\ \hline
{1 server refresh }&{}&{}&{\$40,000} \\ \hline
\textbf{Total}&\textbf{}&\textbf{}&\textbf{\$16,330,000} \\ \hline
\end{longtable} \normalsize



