\section{Conclusion}\label{sec:conc}
We can comply with the requirements and \gls{NIST} 1.7.1 at the cost outlined in \secref{sec:cost}.

There are a few assumptions explicitly made above which we feel comply with given requirements but did require interpretation. To be explicit:
\begin{itemize}
\item \secref{sec:resp} Assumes embargoed images before release to the collaboration are treated securely. After the embargo is lifted there is no longer a need to secure the images at the higher requirements.
\item \secref{sec:resp} Assumes NIST 1.7.1 also applies to \gls{SLAC} even though NIST.FIPS.200 should be applicable.
\item \secref{sec:2firewalls} Makes an important note about \emph{not encrypting} internal \gls{camera} storage.
\item \secref{sec:2firewalls} Assumes \gls{NIST}.800 documents were written as guidance  they will be noted but we may not always follow all recommendations in all cases.
\item \secref{sec:3delay} Assumes the embargo for commissioning applies to use of encrypted storage and transfers. This potentially implies \gls{NCSA} could not be used for commissioning at all.
\item \secref{sec:2firewalls} and \secref{sec:3delay} Assumes short stays of data on unencrypted machines for processing is ok (it is in line with \gls{NIST}).

\end{itemize}
