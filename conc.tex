\section{Conclusion}\label{sec:conc}
We can comply with the requirements and NIST 1.7.1 at the cost outlined in \secref{sec:cost}.

There are a few assumptions explicitly made above which we feel comply with given requirements but did require interpretation. To be explicit:
\begin{itemize}
\item \secref{sec:resp} Assumes embargoed images before release to the collaboration are treated as CUI. After the embargo is lifted they are no longer CUI.
\item \secref{sec:resp} Assumes NIST 1.7.1 also applies to SLAC even though NIST.FIPS.200 should be applicable.
\item \secref{sec:2firewalls} Makes an important note about \emph{not encrypting} internal camera storage.
\item \secref{sec:2firewalls} Assumes NIST.800 documents were written as guidance  they will be noted but we may not always follow all recommendations in all cases.
\item \secref{sec:3delay} Assumes the 30 day embargo for commissioning applies use of encrypted storage and transfers. This potentially implies NCSA could not be used for commissioning at all.
\item \secref{sec:3delay} Assumes short stays of data on unencrypted machines for processing is ok (it is in line with NIST).

\end{itemize}
