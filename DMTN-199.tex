\documentclass[DM,authoryear,toc]{lsstdoc}
% lsstdoc documentation: https://lsst-texmf.lsst.io/lsstdoc.html
\input{meta}

% Package imports go here.

% Local commands go here.

%If you want glossaries
%\input{aglossary.tex}
%\makeglossaries

\title{CUI Rubin Observatory Data Security Standards  Response}

% Optional subtitle
% \setDocSubtitle{A subtitle}

\author{%
William O'Mullane
}

\setDocRef{DMTN-199}
\setDocUpstreamLocation{\url{https://github.com/lsst-dm/dmtn-199}}

\date{\vcsDate}

% Optional: name of the document's curator
% \setDocCurator{The Curator of this Document}

\setDocAbstract{%
This is a response to the Controlled Unclassified Information (CUI)  document from the agencies.
}

% Change history defined here.
% Order: oldest first.
% Fields: VERSION, DATE, DESCRIPTION, OWNER NAME.
% See LPM-51 for version number policy.
\setDocChangeRecord{%
  \addtohist{1}{YYYY-MM-DD}{Unreleased.}{William O'Mullane}
}


\begin{document}

% Create the title page.
\maketitle
% Frequently for a technote we do not want a title page  uncomment this to remove the title page and changelog.
% use \mkshorttitle to remove the extra pages

% ADD CONTENT HERE
% You can also use the \input command to include several content files.

\appendix
% Include all the relevant bib files.
% https://lsst-texmf.lsst.io/lsstdoc.html#bibliographies
\section{References} \label{sec:bib}
\renewcommand{\refname}{} % Suppress default Bibliography section
\bibliography{local,lsst,lsst-dm,refs_ads,refs,books}

% Make sure lsst-texmf/bin/generateAcronyms.py is in your path
\section{Acronyms} \label{sec:acronyms}
\addtocounter{table}{-1}
\begin{longtable}{p{0.145\textwidth}p{0.8\textwidth}}\hline
\textbf{Acronym} & \textbf{Description}  \\\hline

AES & Advanced Encryption Standard \\\hline
AP & Alert Production \\\hline
API & Application Programming Interface \\\hline
ASAP & As Soon As Possible \\\hline
AURA & Association of Universities for Research in Astronomy \\\hline
AVS & Alert Vetting System \\\hline
CCD & Charge-Coupled Device \\\hline
CPU & Central Processing Unit \\\hline
CUI & Controlled Unclassified Information \\\hline
ComCam & The commissioning camera is a single-raft, 9-CCD camera that will be installed in LSST during commissioning, before the final camera is ready. \\\hline
DB & DataBase \\\hline
DCR & Differential Chromatic Refraction \\\hline
DIA & Difference Image Analysis \\\hline
DM & Data Management \\\hline
DMS & Data Management Subsystem \\\hline
DMTN & DM Technical Note \\\hline
DOE & Department of Energy \\\hline
DP1 & Data Preview 1 \\\hline
DR & Data Release \\\hline
DRP & Data Release Production \\\hline
Duo & 2 factor authentication system \\\hline
FIPS & Federal Information Processing Standards \\\hline
FITS & Flexible Image Transport System \\\hline
FIU & Florida International University \\\hline
FTE & Full-Time Equivalent \\\hline
FY22 & Financial Year 22 \\\hline
FreeIPA & is an integrated security information management solution \\\hline
GEO & Geosynchronous Earth Orbit \\\hline
IAU & International Astronomical Union \\\hline
IPA & FreeIPA - Identity, Policy, Audit \\\hline
IPsec & Internet Protocol Security \\\hline
IRU & indefinable right to use \\\hline
IT & Information Technology \\\hline
ITTN & IT Technote \\\hline
JSR & Joint Status Review \\\hline
LDM & LSST Data Management (Document Handle) \\\hline
LHN & long haul network \\\hline
LLNL & Lawrence Livermore National Laboratory \\\hline
LPM & LSST Project Management (Document Handle) \\\hline
LSE & LSST Systems Engineering (Document Handle) \\\hline
LSST & Legacy Survey of Space and Time (formerly Large Synoptic Survey Telescope) \\\hline
LSSTPO & LSST Project Office \\\hline
MPC & Minor Planet Center \\\hline
MREFC & Major Research Equipment and Facility Construction \\\hline
NAT & Network Address Translation \\\hline
NCSA & National Center for Supercomputing Applications \\\hline
NIST & National Institute of Standards and Technology (USA) \\\hline
NSF & National Science Foundation \\\hline
NVMe & Non Volatile Memory Express \\\hline
OGA & Other Government Agencies \\\hline
OOB & Out Of Bound (Alternative network access) \\\hline
OSS & Observatory System Specifications; LSE-30 \\\hline
PSF & Point Spread Function \\\hline
QA & Quality Assurance \\\hline
QC & Quality Control \\\hline
RAID & Redundant Array of Inexpensive Disks \\\hline
RDO & Rubin Directors Office \\\hline
ROP & Rubin Operations Plan \\\hline
RSP & Rubin Science Platform \\\hline
RTN & Rubin Technical Note \\\hline
SATA & Serial Advanced Technology Attachment \\\hline
SFR & Star Formation Rate \\\hline
SLAC & SLAC National Accelerator Laboratory \\\hline
SOC & Security Operations Centre \\\hline
SQL & Structured Query Language \\\hline
SQR & SQuARE document handle \\\hline
SRD & LSST Science Requirements; LPM-17 \\\hline
SSD & Solid-State Disk \\\hline
SSID & Service Set Identifier \\\hline
SV & Science Validation \\\hline
TB & TeraByte \\\hline
TLS & Transport Layer Security \\\hline
UK & United Kingdom \\\hline
US & United States \\\hline
USB & Universal Serial Bus \\\hline
USDF & United States Data Facility \\\hline
VLAN &  Virtual Local Area Network \\\hline
VPN & virtual private network \\\hline
\end{longtable}

% If you want glossary uncomment below -- comment out the two lines above
%\printglossaries





\end{document}
