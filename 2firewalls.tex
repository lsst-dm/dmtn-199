
\subsection{Install Firewalls and other physical security devices} \label{sec:2firewalls}

This requirement is for physical and cyber security. It includes installing cameras and locks on racks.
Some of this such as Firewalls is already in the project plan but much of it is not.

Items already in the baseline include:
\begin{itemize}
\item Card access to server rooms.
\item Backup network in case main link fails (though the microwave link is a new addition ..)
\item Auditable process to handle onboarding/offboarding
\item Some cameras are in the project but not complete coverage.

\end{itemize}

The firewalls and physical security will be upgraded to meet the enhanced standard. \tabref{tab:firewalls} includes the items needed for this upgrade.

{\bf Important Note}: We shall ring fence the Camera in its own firewall with more restricted access than the restricted control network.
However we will treat it as a black box deliverable for this requirement. We shall not expect encryption of the internal disks of the camera system. Any perturbation to the camera system will have a deleterious effect on the camera with significant development and schedule impacts.

Signage and labeling, as required in NIST 171 3.8.4 \footnote{\url{https://www.archives.gov/files/cui/20161206-cui-marking-handbook-v1-1.pdf}}, will be developed as appropriate.

NIST 1.7.1 Section 3.10.6 pulls in extra standards for remote work namely \citeds{nist800-46} and \citeds{nist800-114}.
\citeds{nist800-114} is the broader scope and we are pretty much in line with how it is written - we note Section 5.2.1 that we use Onepassword as a vault for IT passwords - not paper in a fire proof safe as recommended.
Some other suggestions are understood to be useful in general but often not suitable for developers - personal firewalls, application filtering  and aggressive antivirus software often trip over developer code and tools.

\citeds{nist800-46} and other related NIST documentation suggest threat modeling - we do this in a limited way e.g \citeds{SQR-041} and \citeds{SQR-037}. A more exhaustive risk assessment by a third party is not anticipate at this time but the Project team will discuss with SLAC on any plans to review the USDF.
We do not store sensitive information on the VPN nor bastion nodes.
We do use NAT in a limited number of places - this will be more important in operations if/when we move to IPv6.



\tiny \begin{longtable} {|p{0.3\textwidth}|r|r|r|} \caption{This table provides cost estimates for firewalls and other physical security in Chile and at SLAC not in the project plan. \label{tab:firewalls}}\\ 
\hline 
\textbf{Item}&\textbf{Cost}&\textbf{number}&\textbf{Total} \\ \hline
{Locks SLAC}&{\$13}&{30}&{\$390} \\ \hline
{Cameras Detectors  SLAC}&{\$2,000}&{1}&{\$2,000} \\ \hline
{Sensors SLAC}&{\$38}&{30}&{\$1,140} \\ \hline
{Sensor hub SLAC}&{\$448}&{1}&{\$448} \\ \hline
{Locks Chile}&{\$13}&{20}&{\$260} \\ \hline
{Cameras Detectors Chile}&{\$2,000}&{2}&{\$4,000} \\ \hline
{Sensors Chile}&{\$38}&{20}&{\$760} \\ \hline
{Sensor hub Chile}&{\$448}&{2}&{\$896} \\ \hline
{Faster CPU to handle disk encryption on summit (node price)}&{\$13,000}&{20}&{\$260,000} \\ \hline
{SSD price differnce to SATA (cost/TB)}&{\$250}&{260}&{\$65,000} \\ \hline
{Labor to redeploy all summit systems (contract)}&{\$100}&{1,200}&{\$120,000} \\ \hline
{Labelling and signage (CUI)}&{\$2,000}&{1}&{\$2,000} \\ \hline
{Security related contracts/month}&{\$40,000}&{24}&{\$960,000} \\ \hline
{Operations Security contracts}&{\$40,000}&{120}&{\$4,800,000} \\ \hline
\textbf{Total Construction}&\textbf{}&\textbf{}&\textbf{\$1,416,894} \\ \hline
\textbf{Total Operations}&\textbf{}&\textbf{}&\textbf{\$4,800,000} \\ \hline
\end{longtable} \normalsize




This enhanced security plan includes support from an outside security provider.
It is estimated running an SOC could cost upward of \$1.4M per year\footnote{\url{https://expel.io/blog/how-much-does-it-cost-to-build-a-24x7-soc/}}.
This article\footnote{\url{https://www.linkbynet.com/outsourced-soc-vs-internal-soc-how-to-choose}} outlines the pros and cons of
an outsourced SOC and estimates it at between 300 and 800K per year.
For budgeting purposes  \$40K a month is included in \tabref{tab:firewalls}.
Such a contract (or contracts) should cover:

\begin{enumerate}
\item Proactive monitoring and alerting (NIST 171 section 3.3.5)
  \begin{itemize}
  \item Write alerts for suspicious behaviors
  \item Analyze collected logs for anomalies
  \end{itemize}
\item Root cause analysis of any alert or anomaly
\item Incident response
  \begin{itemize}
  \item Isolation of attacker
  \item Forensic analysis leading to timeline and inventory of compromise
  \item Identifying systems that will need to be rebuilt
  \end{itemize}
\item Vulnerability scanning including filtering out false positives
\item Asset inventory including patch status
\item Penetration testing to proactively look for vulnerabilities
\end{enumerate}

This will require extensive coordination and integration with existing IT
services and processes, included as part of this cost.

Since we will have to encrypt systems on the summit (see \citeds{ITTN-014}) for a list of systems)
we anticipate upgrade processors and solid state drives (SSD) are required. Determining the detailed specifications will require experimentation so the
values in the table for this are engineering estimates.

Note that compute facilities for the Commissioning Cluster at the Base as well as Alert Production and the Staff RSP at the USDF are not considered to be within the physical security area.
Rubin considers the short-term, ephemeral processing on these resources outside of the enhanced security requirmenets. Including them would approximately double the cost of this item for Construction.
