
\subsection{Install Firewalls and other physical security devices} \label{sec:2firewalls}

This requirement is for physical and cyber security. It includes installing cameras and locks on racks.
Some of this such as Firewalls is already in the project plan but much of it is not.

Items already in the plan:
\begin{itemize}
\item Card access to server rooms.
\item Backup network in case main link fails (though the microwave link is a new addition ..)
\item Auditable process to handle onboarding/offboarding
\item Some cameras are in the project but not complete coverage.

\end{itemize}

We will do as requested and cost estimates are provided in \tabref{tab:firewalls}.

{\bf Important Note}: We shall ring fence the Camera in its own firewall with more restricted access than the restricted control network.
However we will treat it as a black box deliverable for this requirement. We shall not expect encryption of the internal disks of the camera system. Any perturbation to the camera system tends to extend the project baseline.

I am not sure how to cost signage and labeling as required in NIST 171 3.8.4 \footnote{\url{https://www.archives.gov/files/cui/20161206-cui-marking-handbook-v1-1.pdf}}

NIST 1.7.1 Section 3.10.6 pulls in extra standards for remote work namely \citeds{NIST.800-46} and \citeds{NIST.800-114}.
\citeds{NIST.800-114} is the broader scope and we are pretty much in line with how it is written - we note Section 5.2.1 that we use Onepassword as a vault for IT passwords - not paper in a fire proof safe as recommended.
Some other suggestions are understood to be useful in general but often not suitable for developers - personal firewalls, application filtering  and aggressive antivirus software often trip over developer code and tools.

Since these documents were written  as guidance we will take note of them but may not always follow all recommendations in all cases.
\citeds{NIST.800-46} and other NIST documentation suggest threat modeling - we do this in a limited way e.g \citeds{SQR-041} and \citeds{SQR-037}. A more exhaustive risk assessment may be best done by a third party. We should discuss with SLAC.
We do not store sensitive information on the VPN nor bastion nodes.
We do use NAT in a limited number of places - this will be more important in operations if/when we move to IPv6.


We need to cost security contracts, its a hot area and can e expensive. A nominal \$10K a month is put in \tabref{tab:firewalls}.

\tiny \begin{longtable} {|p{0.3\textwidth}|r|r|r|} \caption{This table provides cost estimates for firewalls and other physical security in Chile and at SLAC not in the project plan. \label{tab:firewalls}}\\ 
\hline 
\textbf{Item}&\textbf{Cost}&\textbf{number}&\textbf{Total} \\ \hline
{Locks SLAC}&{\$13}&{30}&{\$390} \\ \hline
{Cameras Detectors  SLAC}&{\$2,000}&{1}&{\$2,000} \\ \hline
{Sensors SLAC}&{\$38}&{30}&{\$1,140} \\ \hline
{Sensor hub SLAC}&{\$448}&{1}&{\$448} \\ \hline
{Locks Chile}&{\$13}&{20}&{\$260} \\ \hline
{Cameras Detectors Chile}&{\$2,000}&{2}&{\$4,000} \\ \hline
{Sensors Chile}&{\$38}&{20}&{\$760} \\ \hline
{Sensor hub Chile}&{\$448}&{2}&{\$896} \\ \hline
{Faster CPU to handle disk encryption on summit (node price)}&{\$13,000}&{20}&{\$260,000} \\ \hline
{SSD price differnce to SATA (cost/TB)}&{\$250}&{260}&{\$65,000} \\ \hline
{Labor to redeploy all summit systems (contract)}&{\$100}&{1,200}&{\$120,000} \\ \hline
{Labelling and signage (CUI)}&{\$2,000}&{1}&{\$2,000} \\ \hline
{Security related contracts/month}&{\$40,000}&{24}&{\$960,000} \\ \hline
{Operations Security contracts}&{\$40,000}&{120}&{\$4,800,000} \\ \hline
\textbf{Total Construction}&\textbf{}&\textbf{}&\textbf{\$1,416,894} \\ \hline
\textbf{Total Operations}&\textbf{}&\textbf{}&\textbf{\$4,800,000} \\ \hline
\end{longtable} \normalsize




Some open points which may require contracts:
\begin{enumerate}
\item Exhaustive threat modeling - and keeping it up to date.
\begin{itemize}
\item Including Detection, Analysis, Action modeling.
\end{itemize}
\item Log analysis
\item Vulnerability analysis/scanning
\item NIST 171 section  3.3.5 Audit log correlation
\item Developer oriented training exercises and awareness
\end{enumerate}
