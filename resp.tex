\section{Response to the requirements}\label{sec:resp}

There is an implication that we should follow \citeds{NIST.SP.800-171}, as for any standard that is open to some interpretation.
We will have to show how we comply to the standard.
This may take the form of a compliance matrix as shown in \appref{sec:compliance}.
In this matrix and in this document we assume CUI refers to embargoed images before release to the collaboration.
Hence it applies to Prompt Processing, the embargoed data store(s) and the summit in Chile. It does snot apply to DACs nor the actual alert stream.

We note SLAC should comply with \citeds{FIPS200}, FIPS.99, 800-53 and 800-60 as a Federal agency.
We assume our NIST 800-171  will also apply to SLAC since NIST 800-171 is derived from exactly these documents.

From Section 2.1 of \citeds{NIST.SP.800-171} we note the The confidentiality impact value for the data  is no less than moderate.
So we may assume our \citeds{FIPS200} security category would be \{ moderate, low, low\}\footnote{\{confidentiality, availability, integrity\}}.


\subsection{Encrypt Data} \label{sec:1encrypt}

As outlined in \citeds{DMTN-108}  we propose to buy fuor routers which can perform AES IPSec 256 bit encryption between Chile and SLAC.
We will not transfer embargoed images to France - hence we should keep an secure data store at Chile and at SLAC for redundancy.

See \tabref{tab:ipsec} for the cost breakdown.

\tiny \begin{longtable} {|l|r|r|r|} \caption{This table provides cost estimates for encrypted data transfer. \label{tab:ipsec}}\\ 
\hline 
\textbf{Item}&\textbf{Cost}&\textbf{number}&\textbf{Total} \\ \hline
{Cisco Router }&{\$800,000.00}&{4}&{\$3,200,000.00} \\ \hline
{Cabling}&{\$1,000.00}&{4}&{\$4,000.00} \\ \hline
{Misc}&&& \\ \hline
\textbf{Total}&\textbf{}&\textbf{}&\textbf{\$3,204,000.00} \\ \hline
\end{longtable} \normalsize




\subsection{Install Firewalls and other physical security devices} \label{sec:2firewalls}

This requirement is for physical and cyber security. It includes installing cameras and locks on racks.
Some of this such as Firewalls is already in the project plan but much of it is not.

Items already in the plan:
\begin{itemize}
\item Card access to server rooms.
\item Backup network in case main link fails (though the microwave link is a new addition ..)
\item Auditable process to handle onboarding/offboarding
\item Some cameras are in the project but not complete coverage.

\end{itemize}

We will do as requested and cost estimates are provided in \tabref{tab:firewalls}.

{\bf Important Note}: We shall ring fence the Camera in its own firewall with more restricted access than the restricted control network.
However we will treat it as a black box deliverable for this requirement. We shall not expect encryption of the internal disks of the camera system. Any perturbation to the camera system tends to extend the project baseline.

I am not sure how to cost signage and labeling as required in NIST 171 3.8.4 \footnote{\url{https://www.archives.gov/files/cui/20161206-cui-marking-handbook-v1-1.pdf}}

NIST 1.7.1 Section 3.10.6 pulls in extra standards for remote work namely \citeds{NIST.800-46} and \citeds{NIST.800-114}.
\citeds{NIST.800-114} is the broader scope and we are pretty much in line with how it is written - we note Section 5.2.1 that we use Onepassword as a vault for IT passwords - not paper in a fire proof safe as recommended.
Some other suggestions are understood to be useful in general but often not suitable for developers - personal firewalls, application filtering  and aggressive antivirus software often trip over developer code and tools.

Since these documents were written  as guidance we will take note of them but may not always follow all recommendations in all cases.
\citeds{NIST.800-46} and other NIST documentation suggest threat modeling - we do this in a limited way e.g \citeds{SQR-041} and \citeds{SQR-037}. A more exhaustive risk assessment may be best done by a third party. We should discuss with SLAC.
We do not store sensitive information on the VPN nor bastion nodes.




\tiny \begin{longtable} {|p{0.2\textwidth}|r|r|r|} \caption{This table provides cost estimates for firewalls and other physical security in Chile and at SLAC not in the project plan. \label{tab:firewalls}}\\ 
\hline 
\textbf{Item}&\textbf{Cost}&\textbf{number}&\textbf{Total} \\ \hline
{Locks SLAC}&{\$13}&{\$30}&{\$390} \\ \hline
{Cameras Detectors  SLAC}&{\$2,000}&{\$1}&{\$2,000} \\ \hline
{Sensors SLAC}&{\$38}&{\$30}&{\$1,140} \\ \hline
{Sensor hub SLAC}&{\$448}&{\$1}&{\$448} \\ \hline
{Locks Chile}&{\$13}&{\$20}&{\$260} \\ \hline
{Cameras Detectors Chile}&{\$2,000}&{\$2}&{\$4,000} \\ \hline
{Sensors Chile}&{\$38}&{\$20}&{\$760} \\ \hline
{Sensor hub Chile}&{\$448}&{\$2}&{\$896} \\ \hline
{Faster CPU to handle disk encryption on summit}&{}&{}&{\$0} \\ \hline
{Labor to redeploy all summit systems}&{\$100}&{\$160}&{\$16,000} \\ \hline
\textbf{Total }&\textbf{}&\textbf{}&\textbf{\$25,894} \\ \hline
\end{longtable} \normalsize




Some open points which may require contracts:
\begin{enumerate}
\item Exhaustive threat modeling - and keeping it up to date.
\item Log analysis
\item Vulnerability analyst
\end{enumerate}


\subsection{Delay public release} \label{sec:3delay}

The best approach here is to keep the embargoed data on a secure device separate from other systems and migrate images to the regular repository as they become \emph{public}.
This can be an object store with encryption like MinIO \footnote{\url{ https://min.io/product/enterprise-object-storage-encryption}}.
We will need to have one at SLAC and one at Chile for redundancy to ensure no data loss.

With the commissioning constraint that means this needs to be a 30 day store  for Full images and engineering data looking at \citeds{DMTN-135}
table table 40 this comes out to about 500TB of usable disk.
\tabref{tab:delay} gives the cost calculation or this.

\tiny \begin{longtable} {|p{0.2\textwidth}|r|l|} \caption{This table provides costs for the embargoed data store.  \label{tab:delay}}\\ 
\hline 
\textbf{Number of days data to store}& \\ \hline
{Raw data size per day (TB compressed)}& \\ \hline
{Useable size needed (TB)}& \\ \hline
{Allowing for RAID (TB)}& \\ \hline
{Cost for 1 store}& \\ \hline
\textbf{Total for 2 stores}& \\ \hline
\end{longtable} \normalsize



{\bfi Note:} If we assume the security requirements apply to commissioning i.e. we must use encrypted storage and encrypted lines, then we can not carry out commissioning activities at NCSA. This implies SLAC must be ready for ComCam.

\figref{fig:arch} depicts the encrypted storage and network. Embargoed (delayed) data would be held in this encrypted stores for the time specified.
We assume temporary processing for alerts does not have to encrypted, NIST allows ephemeral unencrypted CUI for processing.

\begin{figure}
\begin{centering}
\includegraphics[width=0.9\textwidth]{OGA_Diagram}
	\caption{ OGA architecture  showing the short term encrypted storage and encrypted network from Chile to SLAC. \label{fig:arch}}
\end{centering}
\end{figure}


\subsection{Eliminate earth orbiting satellites} \label{sec:4eliminate}

Rubin does not publish alerts for streaks.
A subset of streaks, potentially consistent with Earth-orbiting satellites or Solar System objects, will be evaluated by the AVS.
AVS is under discussion currently in terms of design and how it may be implemented.
The cost here is mainly FTE related the current OPS plan contains 2.5 FTE for this work.
There is an unknown hardware aspect here - assuming a database already exists a fast front end server will still be needed with
some redundancy.
An estimate is given in \tabref{tab:eliminate}.

\tiny \begin{longtable} {|l|r|r|r|} \caption{The Alert Vetting System is all FTE cost - apart from unknown hardware at LLNL. \label{tab:eliminate}}\\ 
\hline 
\textbf{Description}&\textbf{Cost}&\textbf{Count}&\textbf{Total} \\ \hline
{FTE per year}&{\$500,000.00}&{2.5}&{\$1,250,000} \\ \hline
{Mission years}&{}&{10}&{\$12,500,000} \\ \hline
{Pre operations years}&{}&{3}&{\$3,750,000} \\ \hline
{Front end server}&{\$20,000.00}&{2}&{\$40,000} \\ \hline
{1 server refresh }&{}&{}&{\$40,000} \\ \hline
\textbf{Total}&\textbf{}&\textbf{}&\textbf{\$16,330,000} \\ \hline
\end{longtable} \normalsize





\subsection{Perform earth orbiting satellite processing in separate facility} \label{sec:5perform}
This is under discussion with LLNL - initial cost estimates are given in \secref{sec:4eliminate}.

\subsection{Publish nominal schedule} \label{sec:6publish}


\subsection{Request approval for non sidereal tracking} \label{sec:7request}

This is best handled prropcedurally and as such will not produce a delta cost on the project.

