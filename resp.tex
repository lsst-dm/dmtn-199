\section{Response to the requirements}\label{sec:resp}

There is an implication that we should follow \citeds{NIST.SP.800-171}. As for any standard that is open to some interpretation.
We will have to show how we comply to the standard.
This may take the form of a compliance matrix as shown in \appref{sec:compliance}.
In this matrix and in this document we assume the requirements apply to embargoed images before release to the collaboration and the derived difference image sources.
Hence it applies to Prompt Processing, the embargoed data store(s), and the summit in Chile. It does not apply to DACs nor the actual alert stream.

We note SLAC should comply with \citeds{FIPS200}, FIPS.99, 800-53 and 800-60 as a Federal agency.
We assume our NIST 800-171  will also apply to SLAC since NIST 800-171 is derived from exactly these documents.

From Section 2.1 of \citeds{NIST.SP.800-171} we note that the confidentiality impact value for the data  is no less than moderate.
So we may assume our \citeds{FIPS200} security category would be \{ moderate, low, low\}\footnote{\{confidentiality, availability, integrity\}}.


\subsection{Encrypt Data} \label{sec:1encrypt}

As outlined in \citeds{DMTN-108}  we propose to buy four routers which can perform AES IPSec 256 bit encryption between Chile and SLAC.
We will not transfer embargoed images to France - hence we should keep a secure data store at Chile and at SLAC for redundancy.
The router cost in \tabref{tab:ipsec}  is based on a quotation from Cisco as one of the vendors explicitly specified in the agency document.

NIST also suggests out of band access - an independent network for alerts in case the main network is down.
A quote for Telconor to give a backup control link is included in \tabref{tab:ipsec}.

See \tabref{tab:ipsec} for the cost breakdown.  The OOB access is in Chile only and the routers and cabling are an even split.

\tiny \begin{longtable} {|l|r|r|r|} \caption{This table provides cost estimates for encrypted data transfer. \label{tab:ipsec}}\\ 
\hline 
\textbf{Item}&\textbf{Cost}&\textbf{number}&\textbf{Total} \\ \hline
{Cisco Router (2@Chile 2@SLAC)}&{\$800,000}&{4}&{\$3,200,000} \\ \hline
{Cabling}&{\$1,000}&{4}&{\$4,000} \\ \hline
{Out of Bounds (OOB) link  install (Chile)}&{\$60,000}&{2}&{\$120,000} \\ \hline
{HuaWei Aviod IRU}&{\$2,390,000}&{1}&{\$2,390,000} \\ \hline
{HuaWei Aviod installation}&{\$600,000}&{1}&{\$600,000} \\ \hline
{HuaWei Aviod equipment}&{\$117,000}&{1}&{\$117,000} \\ \hline
{HuaWei Avoid FIU overhead}&{\$11,250}&{1}&{\$11,250} \\ \hline
\textbf{Total Construction}&\textbf{}&\textbf{}&\textbf{\$6,442,250} \\ \hline
{OOB Ops running cost/month}&{\$3,000}&{240}&{\$720,000} \\ \hline
{Router rehresh}&{}&{ }&{\$3,200,000} \\ \hline
{Cabling}&{\$1,000}&{4}&{\$4,000} \\ \hline
{HuaWei Aviod Maint 3\%x10 years}&{\$71,700}&{10}&{\$717,000} \\ \hline
\textbf{Total Operations}&\textbf{}&\textbf{}&\textbf{\$4,641,000} \\ \hline
\end{longtable} \normalsize




\subsection{Install Firewalls and other physical security devices} \label{sec:2firewalls}

This requirement is for physical and cyber security. It includes installing cameras and locks on racks.
Some of this such as Firewalls is already in the project plan but much of it is not.

Items already in the plan:
\begin{itemize}
\item Card access to server rooms.
\item Backup network in case main link fails (though the microwave link is a new addition ..)
\item Auditable process to handle onboarding/offboarding
\item Some cameras are in the project but not complete coverage.

\end{itemize}

We will do as requested and cost estimates are provided in \tabref{tab:firewalls}.

{\bf Important Note}: We shall ring fence the Camera in its own firewall with more restricted access than the restricted control network.
However we will treat it as a black box deliverable for this requirement. We shall not expect encryption of the internal disks of the camera system. Any perturbation to the camera system tends to extend the project baseline.

I am not sure how to cost signage and labeling as required in NIST 171 3.8.4 \footnote{\url{https://www.archives.gov/files/cui/20161206-cui-marking-handbook-v1-1.pdf}}

\tiny \begin{longtable} {|p{0.3\textwidth}|r|r|r|} \caption{This table provides cost estimates for firewalls and other physical security in Chile and at SLAC not in the project plan. \label{tab:firewalls}}\\ 
\hline 
\textbf{Item}&\textbf{Cost}&\textbf{number}&\textbf{Total} \\ \hline
{Locks SLAC}&{\$13}&{30}&{\$390} \\ \hline
{Cameras Detectors  SLAC}&{\$2,000}&{1}&{\$2,000} \\ \hline
{Sensors SLAC}&{\$38}&{30}&{\$1,140} \\ \hline
{Sensor hub SLAC}&{\$448}&{1}&{\$448} \\ \hline
{Locks Chile}&{\$13}&{20}&{\$260} \\ \hline
{Cameras Detectors Chile}&{\$2,000}&{2}&{\$4,000} \\ \hline
{Sensors Chile}&{\$38}&{20}&{\$760} \\ \hline
{Sensor hub Chile}&{\$448}&{2}&{\$896} \\ \hline
{Faster CPU to handle disk encryption on summit (node price)}&{\$13,000}&{20}&{\$260,000} \\ \hline
{SSD price differnce to SATA (cost/TB)}&{\$250}&{260}&{\$65,000} \\ \hline
{Labor to redeploy all summit systems (contract)}&{\$100}&{1,200}&{\$120,000} \\ \hline
{Labelling and signage (CUI)}&{\$2,000}&{1}&{\$2,000} \\ \hline
{Security related contracts/month}&{\$40,000}&{24}&{\$960,000} \\ \hline
{Operations Security contracts}&{\$40,000}&{120}&{\$4,800,000} \\ \hline
\textbf{Total Construction}&\textbf{}&\textbf{}&\textbf{\$1,416,894} \\ \hline
\textbf{Total Operations}&\textbf{}&\textbf{}&\textbf{\$4,800,000} \\ \hline
\end{longtable} \normalsize



\subsection{Delay public release} \label{sec:3delay}

The best approach here is to keep the embargoed data on a secure device separate from other systems and migrate images to the regular repository as they become \emph{public}.
This can be an object store with encryption like MinIO \footnote{\url{ https://min.io/product/enterprise-object-storage-encryption}}.
We will need to have one at SLAC and one at Chile for redundancy to ensure no data loss.

With the commissioning constraint that means this needs to be a 30 day store  for Full images and engineering data looking at \citeds{DMTN-135}
table table 40 this comes out to about 500TB of usable disk.
\tabref{tab:delay} gives the cost calculation or this.

\tiny \begin{longtable} {|p{0.3\textwidth}|r|l|} \caption{This table provides costs for the embargoed data store.  \label{tab:delay}}\\ 
\hline 
\textbf{Description}&\textbf{value}& \\ \hline
{Number of days data to store}&{30}& \\ \hline
{Raw data size per day (TB compressed)}&{16}&{Years data from Table 40 of \citeds{DMTN-135}\/ 298.3 observing nights (Key Numbers Confluence) } \\ \hline
{Useable size needed (TB)}&{484}& \\ \hline
{Allowing for RAID (TB)}&{1000}& \\ \hline
{Cost for 1 store}&{\$400,000}&{Using SLAC Fast Disk Price from Table 28 of \citeds{DMTN-135}} \\ \hline
\textbf{Total for 2 stores}&\textbf{\$800,000}& \\ \hline
{Ops Cost at least 1 Refresh}&{\$800,000}& \\ \hline
\end{longtable} \normalsize



\bf{Note:} If we assume the security requirements apply to commissioning i.e. we must use encrypted storage and encrypted lines, then we can not carry out commissioning activities at NCSA. This implies SLAC must be ready for ComCam.


\subsection{Eliminate earth orbiting satellites} \label{sec:4eliminate}


\subsection{Perform earth orbiting satellite processing in separate facility} \label{sec:5perform}
This is under discussion with LLNL - initial cost estimates are given in \secref{sec:4eliminate}.
It is shown in \figref{fig:arch}.

\subsection{Publish nominal schedule} \label{sec:6publish}

The project was already planning to publish the observing schedule to allow co observing of sources, see Section 2.1 of \citeds{LSE-30}.
The \gls{OSS} requires  publication at least two  hours ahead of observing - the request here is to have the schedule twenty four hours in advance.
This is not a problem as long as one understands the fidelity of the schedule decreases with the look ahead time.
The agency requirement acknowledges this.

The schedule is to be delivered to the trusted broker - we shall arrange this with \gls{LLNL}.

We consider no delta cost for this as it was in the project plan.



\subsection{Request approval for non sidereal tracking} \label{sec:7request}

