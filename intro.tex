\section{Introduction}

The agencies have provided a set of requirements for data security which are addressed in this upgrade plan.  This document addresses the upgrades specifically and will augment the overall security plan for Rubin observatory (see \citeds{LDM-324}).

The requirements can be organized and summarized at a high level as follows:
\begin{enumerate}

\item	Encrypt data following \gls{NIST} .
\item	Use firewalls to prevent unauthorized  access.
\item	Hold focal plane scientific data for a few days following the observation. Hold engineering and commissioning imaging data for several weeks.
\item	Do not alert on artificial Earth-orbiting satellites.
In order to do this, check some alerts against appropriate satellite catalogs.
\item	Publish the survey schedule 24 hours in advance.
\item	Request approval to observe without sidereal tracking.

\end{enumerate}

\secref{sec:resp} provides a subsection response for each of these bullets.

As we approach the operations phase of Rubin Observatory and prepare to deliver data products to the community for the Legacy Survey of Space and Time, we want to remind the community of the basic data products delivered by alert production and data releases.

Nightly alert packets will be produced and streamed to community brokers at the 60 sec cadence (design specification, the minimum requirement is 120 sec), including postage stamps of size $32 \times 32$ pixels ($6.4 \times 6.4$ sq.\ arcsec, or larger---for more details, please see ls.st/dpdd). 
A database of objects detected on difference images will be updated nightly and available through the Rubin Science Platform (RSP). 
Following guidance from the funding agencies, full frame (3.2 Gpix) images in all three flavors (raw, calibrated, and difference), will be made available 80 hours after they were obtained.
A small fraction of the total of all images obtained may be held longer than 80 hr. 
Rubin has consulted with scientists in the community and determined that the policy directive for this 80 hour delay will not significantly impact most science investigations.

In addition, commissioning data will be embargoed for all non-commissioning team staff for 30 days. 
Beyond this embargo, no proprietary data products from commissioning may be shared outside the Commissioning Team prior to the associated data release without explicit approval (ls.st/sitcomtn-010). 
Commissioning Team members are expected to use approved Project tools and processes for communication, data access and analysis, documentation, software development, work management, etc.
In practice, we expect almost all work done by the Commissioning Team on the commissioning data products to be done within private directories at one of two Rubin Science Platform instances, one at the Rubin Us Data Facility at SLAC, and the other at the Rubin Interim Data Facility in the Google cloud.
As per \citeds{SITCOMTN-010}, the Project will define a process to approve the sharing of {\it derived data products} (see the Rubin Data Policy, \citeds{RDO-013}) based on commissioning data prior to the associated data release.
This is needed to enable the LSST science community to begin learning about Rubin data in preparation for their survey analyses. 

Appropriate data and data products from commissioning will be assembled in data previews  and released to the community within about 6 months of the end of the associated phase of commissioning (e.g, end of ComCam use plus 6 months).
It is expected that a data preview based on science validation survey data from the ultimate phase of commissioning (about 2 months of Science Validation Surveys) will be made available early in full survey operations, before the first data release processing begins.
Survey data releases including coadded images and updated source photometry will begin about one year after the start of the full survey. 
Our current plan is to acquire six months of data for the first release.
Alert production will reach its full throughput and steady-state efficiency after Data Release 1; during the first survey year, alerts will be produced incrementally once adequate image templates are available (for more details, please see ls.st/dmtn-107). 
