\section{Introduction}

The agencies have provided a set of requirements for data security which are addressed in this upgrade plan.  This document addresses the upgrades specifically and will augment the overall security plan for Rubin observatory (see \citeds{LDM-324}).

The requirements can be organized and summarized at a high level as follows :
\begin{enumerate}

\item	Encrypt data following \gls{NIST} .
\item	Use firewalls to prevent unauthorized  access.
\item	Hold focal plane scientific data for a few days following the observation. Hold engineering and commissioning imaging data for several weeks.
\item	Do not alert on artificial Earth-orbiting satellites.
In order to do this, check some alerts against appropriate satellite catalogs.
\item	Publish the survey schedule 24 hours in advance.
\item	Request approval to observe without sidereal tracking.

\end{enumerate}


\secref{sec:resp} provides a subsection response for each of these bullets.

As we approach the operations phase of Rubin Observatory and prepare to deliver data products to
the community for the Legacy Survey of Space and Time, we want to remind the community of the
basic data products delivered by alert production and data releases.

Nightly alert packets will be produced and streamed to community brokers at the 60 sec cadence
(design specification, the minimum requirement is 120 sec), including postage stamps of size
32 x 32 pixels (6.4 x 6.4 sq. arcsec, or larger, for more details, please see ls.st/dpdd). A database of
objects detected on difference images will be updated nightly and available through the Rubin Science
Platform (RSP). Following guidance from the funding agencies, full frame (3.2 Gpix) images in all
three flavors (raw, calibrated, and difference), will be made available 80 hours after they were obtained.
A small fraction of the total of all images obtained may be held longer than 80 hr. Rubin has consulted
with scientists in the community and determined that the policy directive for this 80 hour delay will not
significantly impact most science investigations.

Data releases including coadded images and updated source photometry will begin about one year
after the start of the full survey. Our current plan is to acquire six months of data for the first release.
Alert production will reach its full throughput and steady-state efficiency after Data Release 1; during
the first survey year, alerts will be produced incrementally once adequate image templates are
available (for more details, please see ls.st/dmtn-107). It is expected that a data preview based on
science validation survey data from the ultimate phase of commissioning (about 2 months of Science
Validation Surveys) will be made available early in full survey operations, before the first data release
processing begins.
