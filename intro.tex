\section{Introduction}

The funding agencies, \gls{NSF} and \gls{DOE},  have provided a set of requirements for data security which are addressed in this upgrade plan.  This document addresses the upgrades specifically and will augment the overall security plan for Rubin observatory (see \citeds{LDM-324}).

The requirements can be organized and summarized at a high level as follows:
\begin{enumerate}

\item	Encrypt data using strong, approved encryption standards,  following \citeds{NIST.SP.800-171} for \gls{CUI} at non federal organizations. The required use of these \gls{NIST} security standards is limited to the physical security and encryption points. It does not extend to treating the data as \gls{CUI}. The data should \emph{not} be marked as \gls{CUI}.
\item	Use firewalls and physical security best practices to prevent unauthorized network  access. Documented compliance shall be in accordance with \citeds{NIST.SP.800-171}.
\item	Delay public release of  focal plane scientific data for an embargo period of at least 80 hours following the  observation.
Hold engineering mode images for 30 days and survey-mode images (e.g. Science Validation survey images) for the regular survey operations period of 80 hours.
Data aside from focal plane data may be made available following the original project plan which includes astronomical metadata (within 24 hours), standard \gls{postage stamp} images (within 60 sec) not corresponding to artificial  Earth-orbiting satellites (see Requirement 4), and weather and sky monitoring data.
NSF and \gls{DOE} require a system in  place to extend the embargo times for the release of focal plane scientific data in the unlikely event that it is needed.

\item	Eliminate artificial Earth-orbiting satellites.
To do this, do not issue alerts on streaks that correspond to objects  moving faster than 10 \gls{deg}/day relative to sidereal tracking or for objects whose angular velocity cannot be determined.
Additionally, any object in the appropriate catalogs provided to the U.S. Data Facility shall be eliminated from the  Prompt \gls{Alert} Stream and withheld from the publicly searchable Prompt Processing Database.
SLAC shall handle any  observations and/or ephemerides used to create and/or update orbital elements in a satellite database or catalog as  OFFICIAL USE ONLY and implement appropriate controls as directed by \gls{DOE}.
\item	Publish the nominal survey schedule 24 hours and an updated schedule at least 2 hours in advance.
Once complete, publicly provide the actual executed observation.
It is understood the schedule may change in real-time due to  weather or other unforeseen circumstances.

\item	Only observe without sidereal tracking in regions pre-approved by \gls{NSF} and \gls{DOE}.
Currently, the only restrictions when operating in non-sidereal tracking modes include that no part of the field of view shall be within $\pm 2$ degrees  inclination of the \gls{GEO} belt orbital plane.
No \gls{camera} bore sights shall be between $+1.0$ and $+9.0$ degrees of declination.
Currently, no other bore sight restrictions are anticipated; however, \gls{NSF} and \gls{DOE} may add additional non-sidereal tracking field of view restrictions in the future.

\end{enumerate}
Section~\ref{sec:resp} provides a subsection response for each of these bullets.
The six requirements summarized above will be in place for the commissioning phase of the integrated Rubin Observatory system.
Specifically, the requirements will pertain to data obtained with the LSST Camera in place on the telescope for commissioning and going forward into full survey operations. 
The data security requirements in the present document do not apply to earlier phases of Rubin commissioning with the Commissioning Camera (also known as \gls{ComCam}), nor to the data obtained in any phase at the Rubin Auxilliary Telescope (AuxTel).


As we approach the operations phase of Rubin Observatory and prepare to deliver data products to the community for the Legacy Survey of Space and Time, we want to remind the community of the basic data products delivered by alert production and data releases.

Nightly alert packets will be produced and streamed to community brokers at the 60 sec cadence (design specification, the minimum requirement is 120 sec), including postage stamps of size $32 \times 32$ pixels ($6.4 \times 6.4$ sq.\ \gls{arcsec}, or larger---for more details, please see \citeds{LSE-63}).
A database of objects detected on difference images will be updated nightly and available through the Rubin Science Platform (\gls{RSP}).
Following guidance from the funding agencies, full frame (3.2 Gpix) images in all three flavors (raw, calibrated, and difference), will be made available 80 hours after they were obtained.
A small fraction of the total of all images obtained may be held longer than 80 hr and a mechanism for extending embargo periods for certain images at the request of the funding agencies will be implemented.
Rubin has consulted with scientists in the community and determined that the policy directive for this 80 hour delay will not significantly impact most science investigations.
Additionally no prompt alerts will be issued for sources corresponding to objects in an appropriate satellite catalog.
And no prompt alerts for streaks greater than 10 \gls{deg}/day relative to sidereal.
The exclusion of prompt alerts for these streaks will have minimal impact on science.
In the rare event an unknown solar system object source with a streak length greater than 10 \gls{deg}/day relative to sidereal potentially corresponds with an Earth-impacting asteroid, the \gls{USDF} will implement a method to send any candidate impactors to the \gls{MPC} during the embargo period.
All data in transit shall comply with the strong, approved encryption standards outlined for the embargo period.

In addition, commissioning engineering data will be embargoed for all non-commissioning team staff for 30 days.
After this 30 day embargo, only with explicit approval may proprietary data products from commissioning be shared outside the \gls{Commissioning} Team  \citeds{SITCOMTN-010}.

The transition from the 30-day embargo to an 80-hour embargo will happen at System First Light (SFL), and when all federal reviews of Rubin Data Management Standards have been passed. 
NSF and DOE will provide simple email confirmation of this, following the final review outbrief.
(Prior to SFL, LSSTCam commissioning takes place in ``engineering mode,'' with observations executed in custom blocks that do not resemble routine sky surveying. 
After SFL, Rubin will switch to ``survey mode,'' with observations executed in schedule blocks similar to those of the LSST.)  

Commissioning Team members are expected to use approved Project tools and processes for communication, data access and analysis, documentation, \gls{software} development, work management, etc.
In practice, we expect most work done by the \gls{Commissioning} Team on the commissioning data to be done within private directories at the Rubin US Data Facility at SLAC.

Appropriate data and data products from commissioning will be assembled in data previews  and released to the community within about 6 months of the end of the associated phase of commissioning (e.g, DP1 will be released approximately 6 months after the \gls{ComCam} observations are completed).
It is expected that a data preview based on science validation survey data from the ultimate phase of LSST Cam commissioning (about 2 months of Science \gls{Validation} Surveys) will be made available early in full survey operations, before the first data release processing begins.

As per \citeds{SITCOMTN-010}, Rubin will define a process to approve the sharing of {\it derived data products} (see the Rubin Data Policy, \citeds{RDO-013}) based on commissioning data with the data rights holder community prior to the associated data preview release.
This is needed to enable the \gls{LSST} science community to begin learning about Rubin data in preparation for their survey analyses.

Survey data releases including coadded images and updated source photometry will begin about one year after the start of the full survey.
Our current plan is to acquire six months of data for the first formal \gls{LSST} data release.
Alert production will reach its full throughput and steady-state efficiency after \gls{Data Release} 1; during the first survey year, alerts will be produced incrementally once adequate image templates are available (for more details, please see \citeds{DMTN-107}).
