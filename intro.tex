\section{Introduction}

The agencies have provided a set of requirements for security which we asses here and provide initial cost impact analysis for.  This is not the overall security plan for Rubin observatory (see \citeds{LDM-324}).
This document addresses these specific requirements given to us by the agencies.

The summary requirements (from the start of the document) are :
\begin{enumerate}

\item	Encrypt data using strong, approved encryption standard, following NIST 800-171 standard for CUI at non-federal organizations.
\item	Install firewalls to prevent unauthorized network access, guided by NIST 800-171 standard for CUI at non-federal organizations.
\item	Delay public release of focal plane scientific data for at least 80 hours following the observation, with Alert Vetting System allowed to withhold up to 4 images per month for up to 10 days with need only for notification to be given to NSF/DOE. Delay public release of engineering and commissioning imaging data for at least 30 days.
\item	Eliminate artificial Earth-orbiting satellites from prompt alerts by (a) automatically alerting only on streaks corresponding to motions slower than 30 deg/day relative to sidereal tracking, and (b) alerting on longer (faster) streaks only after the Alert Vetting System has determined that the streak does not correspond to an artificial satellite.
\item	Perform Earth-orbiting satellite processing in a separate facility operated by a “trusted broker” that has access to appropriate satellite catalogs.
\item	Publish nominal collection schedules for regular sky survey 24 hours in advance.
\item	Request and receive advance approval of large sky regions for use without sidereal tracking prior to initial on-sky test observations; then, approved regions (for use without sidereal tracking) will be supplied to the Rubin Observatory operations team in advance of their use.

\end{enumerate}


\secref{sec:resp} provides a subsection response for each of these bullets.
